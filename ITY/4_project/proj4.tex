\documentclass[a4paper, 11pt]{article}
\usepackage[utf8]{inputenc}
\usepackage[T1]{fontenc}
\usepackage[top=3cm, text={17cm, 24cm}, left=2cm]{geometry}
\usepackage[czech]{babel}

\begin{document}

\begin{titlepage}
\begin{center}
    \Huge
    \textsc{Vysoké učení technické v~Brně\\
    \huge{Fakulta informačních technologií}}\\
    \vspace{\stretch{0.382}}
    \LARGE{Typografie a publikování\,--\,4. projekt}\\
    \Huge{Bibliografické citace}\\
    \vspace{\stretch{0.618}}
\end{center}
{\Large \today \hfill Michal Blažek}
\end{titlepage}

\section{Typografie}

\subsection{Úvod}

Typografie a \LaTeX~hrají důležitou roli ve světě psaní, sazby a designu dokumentů. Od psacích strojů až po moderní digitální technologie přinášejí do tvorby dokumentů efektivitu a estetiku. Typografie tvoří základní vizuální hierarchii, která řídí čtenářovu pozornost viz~\cite{Kottwitz2011}.

\subsection{Historie a důležitost typografie}

Historie typografie sahá až do doby starověkých civilizací, kdy se písmo vyvíjelo z~rytin na jeskynních stěnách. Důležitost správného použití typografie spočívá v~schopnosti efektivně sdělovat informace a vytvářet jejich čtení příjemné. Typografie také klade velký důraz na kompletnost a správnost informací, proto také udává určitá pravidla pro psaní dokumentů. Podle knihy \uv{Základy typografie: 100 principů pro práci s~písmem} (lze najít v~\cite{Saltz2010}) byla patková písma tradičně používána v~knihách, což přispívalo k~čitelnosti a příjemnosti při čtení textu. Jelikož autoři chtějí, aby čtenáře jejich práce bavila, preferují patková písma. Díky vývoji knihtisku v~15. století byla umožněna obrovská produkce knih. Tento rozmach tisku vedl ke vzniku nových typografických stylů, které ovlivnily celou další historii.

\subsection{Moderní typografie na webových stránkách}

V~moderní době se typografie stala klíčovým prvkem v~designu a to třeba i na webových stránkách. Responzivní design umožňuje flexibilní a dynamické uspořádání textu na různých zařízeních viz~\cite{FontgemGuideForBegginers}. Minimalistický design je čím dál více častějším prvkem v~moderní typografii. Například optimální délka řádku je kolem 50\,-\,75 znaků, aby se uživatel při čtení neztrácel v~textu viz~\cite{FontgemWebDesign}.

\section{\LaTeX : dokonalý nástroj pro akademické dokumenty}

\LaTeX~je sázecí systém založený na programovacím jazyce \TeX . V poslední době se stal nedílnou součástí akademického prostředí. Provotně může \LaTeX~působit složitě, protože uživatel je nucen psát dokument pomocí příkazů. Avšak nakonec stačí naučit se několik málo příkazů a dokumenty vypadají mnohem profesionálněji viz~\cite{Rybicka2003}. Hlavní schopností \LaTeX u je vytvářet profesionálně vypadající dokumenty s~matematickými vzorci, tabulkami a citacemi, což usnadňuje psaní vědeckých prací a jiných technických dokumentů. \LaTeX~také slouží pro formátování textu nebo pro křížové odkazy, což zase velice usnadňuje tvorbu dlouhých dokumentů, jak se můžeme dočíst v~\cite{Gratzer2013}. 

Spojení tradiční typografie s~moderním nástrojem jako je \LaTeX~přináší do psaní a formátování dokumentů nové možnosti viz~\cite{Jiricek2012}. Podle~\cite{Bednar2011} lze díky technickému know-how vytvářet dokumenty velmi rychle, vypadají profesionálně a také jsou efektivní ve sdělování informací.

\section{Závěr}

Typografie a \LaTeX~jsou velmi důležité části při psaní a designu viz~\cite{Huys2023}. Jejich kombinace nám umožňuje vytvářet kvalitní dokumenty, které jsou vizuálně hezké a zároveň sdělují obsah jasně a efektivně viz~\cite{Kafka2017}. Samozřejmě lze pomocí \LaTeX u psát dokumenty velmi rychle.

\newpage

\bibliographystyle{czechiso}
\bibliography{proj4}

\end{document}
