\documentclass[a4paper, twocolumn, 11pt]{article}
\usepackage[utf8]{inputenc}
\usepackage[T1]{fontenc}
\usepackage[top=2.3cm, text={18.3cm, 25.2cm}, left=1.4cm]{geometry}
\usepackage{lmodern}
\usepackage{amssymb}
\usepackage{amsmath}
\usepackage{amsthm}
\usepackage[czech]{babel}

\newtheorem{definition}{Definice}

\begin{document}

\begin{titlepage}
\begin{center}
    \Huge
    \textsc{Vysoké učení technické v~Brně\\[0.5em]
    \huge{Fakulta informačních technologií}}\\
    \vspace{\stretch{0.382}}
    \LARGE{Typografie a publikování\,--\,2. projekt\\[0.6em]
    Sazba dokumentů a matematických výrazů}\\
    \vspace{\stretch{0.618}}
\end{center}
{\Large 2024 \hfill Michal Blažek (xblaze38)}
\end{titlepage}

\section*{Úvod}

V~této úloze si vysázíme titulní stranu a kousek matematického textu, v~němž se vyskytují například Definice~\ref{definice1} nebo rovnice~(\ref{rovnice2}) na straně~\pageref{rovnice2}. Pro vytvoření těchto odkazů používáme kombinace příkazů \verb|\label|, \verb|\ref|, \verb|\eqref| a \verb|\pageref|. Před odkazy patří nezlomitelná mezera. Pro zvýrazňování textu se používají příkazy \verb|\verb| a \verb|\emph|.

Titulní strana je vysázena prostředím \texttt{titlepage} a~nadpis je v~optickém středu s~využitím \emph{přesného} zlatého řezu, který byl probrán na přednášce. Dále jsou na titulní straně čtyři různé velikosti písma a mezi dvojicemi řádků textu je použito řádkování se zadanou relativní velikostí 0,5\,em a 0,6\,em\footnote{Použijte správný typ mezery mezi číslem a jednotkou.}.

\section{Matematický text}

Matematické symboly a výrazy v~plynulém textu jsou v~prostředí \texttt{math}. Definice a věty sázíme v~prostředí definovaném příkazem \verb|\newtheorem| z~balíku \texttt{amsthm}. Tato prostředí obracejí význam \verb|\emph|: uvnitř textu sázeného kurzívou se zvýrazňuje písmem v~základním řezu. Někdy je vhodné použít konstrukci \verb|${}$| nebo \verb|\mbox{}|, která zabrání zalomení (matematického) textu. Pozor také na tvar i sklon řeckých písmen: srovnejte \verb|\epsilon| a \verb|\varepsilon|, \verb|\Xi| a \verb|\varXi|.

\begin{definition}
    \label{definice1}\emph{Konečný přepisovací stroj} neboli \emph{Mealyho automat} je definován jako uspořádaná pětice tvaru $\mathnormal{M=(Q,\Sigma,\Gamma,\delta,q_0)}$, kde:

    \begin{itemize}
        \item Q je konečná množina \emph{stavů,}
        \item $\mathnormal{\Sigma}$ je konečná vstupní \emph{abeceda,}
        \item $\mathnormal{\Gamma}$ je konečná výstupní \emph{abeceda,}
        \item $\mathnormal{\delta:Q\times\Sigma\rightarrow Q\times\Gamma}$ je totální \emph{přechodová funkce,}
        \item $q_0\in Q$ je \emph{počáteční stav.}
    \end{itemize}
\end{definition}

\subsection{Podsekce s~definicí}

Pomocí přechodové funkce $\delta$ zavedeme novou funkci $\delta^*$ pro překlad vstupních slov $\mathnormal{u\in\Sigma^*}$ do výstupních slov $\mathnormal{w\in\Gamma^*}$.

\begin{definition}
    Nechť $\mathnormal{M=(Q,\Sigma,\Gamma,\delta,q_0)}$ je Mealyho automat. \emph{Překládací funkce} $\mathnormal{\delta^*:Q\times\Sigma^*\times\Gamma^*\rightarrow\Gamma^*}$ je pro každý stav $q\in Q$, symbol $\mathnormal{x\in\Sigma}$, slova $\mathnormal{u\in\Sigma^*}$, $\mathnormal{w\in\Gamma^*}$ definována rekurentním předpisem:
    \begin{itemize}
        \item $\delta^*(q,\varepsilon,w)=w$
        \item $\delta^*(q,xu,w)=\delta^*(q',u,wy)$, kde $(q',y)=\delta(q,x)$
    \end{itemize}
\end{definition}

\subsection{Rovnice}

Složitější matematické formule sázíme mimo plynulý text pomocí prostředí \texttt{displaymath}. Lze umístit i více výrazů na jeden řádek, ale pak je třeba tyto vhodně oddělit, například pomocí \verb|\quad|, při dostatku místa i~\verb|\qquad|.
$$g^{a_n}\notin A^{B^n}\qquad y^1_0-\sqrt[5]{x+\sqrt[7]{y}}\qquad x>y^2\geq y^3$$

Velikost závorek a svislých čar je potřeba přizpůsobit jejich obsahu. Velikost lze stanovit explicitně, anebo pomocí \verb|\left| a \verb|\right|. Kombinační čísla sázejte makrem \verb|\binom|.

\begin{equation*}
    \left|\bigcup P\right|=\sum\limits _{\emptyset\neq X\subseteq P}(-1)^{|X|-1}\left|\bigcap X\right|
\end{equation*}

\begin{equation*}
    F_{n+1}=\binom{n}{0}+\binom{n-1}{1}+\binom{n-2}{2}+\cdots+\binom{\lceil\frac{n}{2}\rceil}{\lfloor\frac{n}{2}\rfloor}
\end{equation*}

V~rovnici~(\ref{rovnice1}) jsou tři typy závorek s~různou \emph{explicitně} definovanou velikostí. Obě rovnice mají svisle zarovnaná rovnítka. Použijte k~tomu vhodné prostředí.
\begin{eqnarray}
    \label{rovnice1}\left(\left\{b\otimes[c_1\oplus c_2]\circ a\right\}^{\frac{2}{3}}\right) & = & \mathrm{log}_z x\\
    \label{rovnice2}\int_a^b f(x)\,\mathrm{d}x & = & -\int_b^a f(y)\,\mathrm{d}y
\end{eqnarray}
V~této větě vidíme, jak se vysází proměnná určující limitu v~běžném textu: $\lim_{m\rightarrow\infty} f(m)$. Podobně je to i s~dalšími symboly jako $\bigcup_{N \in \mathcal{M}}N$ či $\sum_{i=1}^m x_i^2$. S~vynucením méně úsporné sazby příkazem \verb|\limits| budou vzorce vysázeny v~podobě $\lim\limits _{m\rightarrow\infty}f(m)$ a $\sum\limits _{i=1}^m x_i^2$.

\section{Matice}

Pro sázení matic se používá prostředí \texttt{array} a závorky s~výškou nastavenou pomocí \verb|\left|, \verb|\right|.
$$D=
\left|
\begin{array}{cccc}
    a_{11} & a_{12} & \cdots & a_{1n}\\
    a_{21} & a_{22} & \cdots & a_{2n}\\
    \vdots & \vdots & \ddots & \vdots\\
    a_{m1} & a_{m2} & \cdots & a_{mn}
\end{array}
\right|
=
\left|
\begin{array}{cc}
    x & y\\
    t & w
\end{array}
\right|
=xw-yt$$

Prostředí \texttt{array} lze úspěšně využít i jinde, například na pravé straně následující rovnosti.

$$\binom{n}{k}=
\left\{
\begin{array}{ll}
    \frac{n!}{k!(n-k)!} & \text{pro } 0\leq k\leq n\\
    0 & \text{jinak}
\end{array}
\right.$$

\end{document}
