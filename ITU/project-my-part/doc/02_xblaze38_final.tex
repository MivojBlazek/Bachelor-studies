\documentclass[a4paper, 11pt]{article}
\usepackage[utf8]{inputenc}
\usepackage[T1]{fontenc}
\usepackage[top=3cm, text={17cm, 24cm}, left=2cm]{geometry}
\usepackage[czech]{babel}
\usepackage{booktabs}
\usepackage{graphicx}
\usepackage{float}

\begin{document}

\begin{titlepage}
\begin{center}
    \Huge
    \textsc{Vysoké učení technické v~Brně\\
    \huge{Fakulta informačních technologií}}\\
    \vspace{\stretch{0.382}}
    \LARGE{Tvorba uživatelských rozhraní}\\
    \Huge{Systém rozdělování rozhodčích českého hokeje}\\
    \vspace{\stretch{0.618}}
\end{center}
{\Large \today \hfill \parbox{5cm}{\raggedleft Michal Blažek\\Matěj Lepeška\\Matyáš Sapík}}
\end{titlepage}

\section{Popis požadované cílové aplikace}

Aplikace bude sloužit klubům, rozhodčím a delegátům pro řízení utkání. Podle
dotazovaných uživatelů z rozhovorů musí být aplikace přehledná a rychlá. Kluby budou
moci přidat zápasy a zobrazit si jednotlivé delegáty či rozhodčí těchto zápasů a jejich
vyúčtování. Kluby mají možnost spravovat svoje hráče. Rozhodčí se budou v aplikaci
přihlašovat na jednotlivé zápasy a z nich vystavovat vyúčtování. Od delegátů rozhodčí
dostávají posudky společně s videi z jejich zápasů. Další prací delegáta je schvalovaní
vyúčtování rozhodčím. Každý uživatel uvidí svůj profil a ostatní profily klubů, rozhodčích a
delegátů.

\section{Rozdělení práce členů týmu na výsledné implementaci}

Jelikož naše aplikace je pro 3 typy uživatelů, rozdělili jsme FE mezi jednotlivé členy týmu
podle těchto uživatelů následovně:
\begin{itemize}
    \item části aplikace z pohledu delegáta\,--\,Michal Blažek,
    \item části aplikace z pohledu rozhodčího\,--\,Matěj Lepeška a
    \item části aplikace z pohledu klubu\,--\,Matyáš Sapík.
\end{itemize}

\section{Testování}

\subsection{Report z testování (Michal Blažek)}

%%%%%%%%%%%%%%%%%%%%%%%%%%%% TODO %%%%%%%%%%%%%%%%%%%%%%%%%%%%

\subsection{Report z testování (Matěj Lepeška)}

%%%%%%%%%%%%%%%%%%%%%%%%%%%% TODO %%%%%%%%%%%%%%%%%%%%%%%%%%%%

\subsection{Report z testování (Matyáš Sapík)}

%%%%%%%%%%%%%%%%%%%%%%%%%%%% TODO %%%%%%%%%%%%%%%%%%%%%%%%%%%%

\end{document}
