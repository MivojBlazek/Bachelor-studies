\documentclass[a4paper, 11pt]{article}
\usepackage[utf8]{inputenc}
\usepackage[T1]{fontenc}
\usepackage[top=3cm, text={17cm, 24cm}, left=2cm]{geometry}
\usepackage[czech]{babel}
\usepackage{booktabs}
\usepackage{graphicx}
\usepackage{float}

\begin{document}

\begin{titlepage}
\begin{center}
    \Huge
    \textsc{Vysoké učení technické v~Brně\\
    \huge{Fakulta informačních technologií}}\\
    \vspace{\stretch{0.382}}
    \LARGE{Tvorba uživatelských rozhraní}\\
    \Huge{Systém rozdělování rozhodčích českého hokeje}\\
    \vspace{\stretch{0.618}}
\end{center}
{\Large \today \hfill \parbox{5cm}{\raggedleft Michal Blažek\\Matěj Lepeška\\Matyáš Sapík}}
\end{titlepage}

\section{Popis požadované cílové aplikace}

Aplikace bude sloužit klubům, rozhodčím a delegátům pro řízení utkání. Podle
dotazovaných uživatelů z~rozhovorů musí být aplikace přehledná a rychlá. Kluby budou
moci přidat zápasy a zobrazit si jednotlivé delegáty či rozhodčí těchto zápasů a jejich
vyúčtování. Kluby mají možnost spravovat svoje hráče. Rozhodčí se budou v~aplikaci
přihlašovat na jednotlivé zápasy a z~nich vystavovat vyúčtování. Od delegátů rozhodčí
dostávají posudky společně s~videi z~jejich zápasů. Další prací delegáta je schvalovaní
vyúčtování rozhodčím. Každý uživatel uvidí svůj profil a ostatní profily klubů, rozhodčích a
delegátů.

\section{Rozdělení práce členů týmu na výsledné implementaci}

Jelikož naše aplikace je pro 3 typy uživatelů, rozdělili jsme FE mezi jednotlivé členy týmu
podle těchto uživatelů následovně:
\begin{itemize}
    \item části aplikace z~pohledu delegáta\,--\,Michal Blažek,
    \item části aplikace z~pohledu rozhodčího\,--\,Matěj Lepeška a
    \item části aplikace z~pohledu klubu\,--\,Matyáš Sapík.
\end{itemize}

\section{Testování}

\subsection{Report z~testování (Michal Blažek)}

Uživatel z~testování byl hokejový rozhodčí ve věku 21 let. Byl to muž a jeho technická zdatnost je na rozumné úrovni. Není pro něj problém hledat informace na webu a proklikávat se jednotlivými stránkami. Určitě mu také nedělalo problém, že je celá aplikace v~angličtině.

Test probíhal celkem plynule, i přestože je to rozhodčí, který viděl aktuální systém pro delegáty jen okrajově, dokázal se vžít do jejich role a postupně projít všechny důležité případy užití, které ho napadly a očekával je od aplikace. Aplikaci ohodnotil jako mnohem lepší než jeho aktuální s~tím, že je uživatelsky přívětivá.

Také ale měl při testování i potíže. Konkrétně se uživatel snažil filtrovat hry pouze pomocí zadávání do vstupních polí a chvilku trvalo, než si všiml, že filtry musí potvrdit tlačítkem \uv{Filter}. Na další problém narazil, když chtěl zadat zprávu ze zápasu pro rozhodčí a nenašel tlačítko na uložení zprávy. Po chvilce to pochopil a dokonce pochválil toto řešení. Uživatel celou dobu testoval pouze na desktopovém zařízení, tudíž asi ze zvyku používal v~prohlížeči šipku zpět namísto implementovaného tlačítka \uv{Back}. Poslední potíže měl uživatel na stránce s~výpisem všech vyúčtování. Zde nevěděl, které již byly potvrzené a které ne, ale hlavně po rozkliknutí jednotlivých vyúčtování neviděl tlačítka na potvrzení a zamítnutí a ani podle ničeho jiného nedokázal říct, jestli je aktuální vyúčtování již potvrzené, či proč vypadá přesně takhle.

Z~testování byl velmi patrný problém ve vyúčtování. Tento problém byl vyřešen přidáním informace kdy a kdo z~delegátů potvrdil aktuálně otevřené vyúčtování, což jednoznačně ukazuje, že dané vyúčtování je již potvrzené.

\subsection{Report z~testování (Matěj Lepeška)}

Pro testování jsem zvolil muže ve věkové skupině 30 let.  Jeho  technická schopnost je dle mého názoru průměrná a jeho pracovní zaměření občas využívá práci s~různými systémy. Testování probíhalo přibližně 30 minut. Dle názoru testujícího došlo ke splnění většiny požadavků. Jako plus testující zmínil rychlost systému a jeho přehlednost. Jako mínus zmínil občasnou potřebu provést moc interakcí se systémem pro splnění určitého úkolu. Jako ponaučení z~testování jsem si odnesl, že je dobré provádět postupné testování, zda některé činnosti nevyžadují moc interakce.

\subsection{Report z~testování (Matyáš Sapík)}

Uživatel se přihlásil do aplikace, zobrazila se mu Board, tam zkoušel klikat na zápasy, ale to se mu nepodařilo, protože tam není co si rozkliknout. Poté si nahoře všimnul navigace a přesměroval se do rozpisu zápasu, tam si vyzkoušel přidávání, úpravu a odebírání zápasů, tam se mu vše líbilo jak bylo. Potom se odnavigoval na stránku se sestavou, kde měl připomínky, že by mohla být sestava na každý zápas jiná. Poté šel do profilu, kde se mu také vše líbilo. Celkově vzhled a celou aplikaci hodnotil celkem pozitivně, jen říkal že by tam mohli být ještě nějaké malé úpravy.

\end{document}
